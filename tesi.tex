\documentclass{article}
\usepackage[italian]{babel}
\usepackage{tikz}
\usetikzlibrary{automata, positioning, arrows}
\tikzset{
    ->, % makes the edges directed
    >=stealth', % makes the arrow heads bold
    node distance=5cm, % specifies the minimum distance between two nodes. Change if necessary.
    every state/.style={thick, fill=gray!10, minimum size=2.5cm}, % sets the properties for each ’state’ node
    initial text=$ $, % sets the text that appears on the start arrow
}

\title{Automazione di test di accettazione per dispositivi IoT embedded integrati nel cloud}
\author{Alessandro Righi}
\date{\today}

\begin{document}
\maketitle
\pagebreak
\tableofcontents
\pagebreak

\section{Introduzione}
\subsection{contesto "IOTINGA"}

IOTINGA s.r.l. è una giovane realtà informatica veronese. Fra i vari settori in cui
opera ricopre particolare rilevanza quello dell'IoT e dei prodotti connessi al cloud.

IRSAP s.p.a. è invece una grossa azienda rodigina leader nel settore del comfort termico.
Storicamente produttrice di radiatori si distingue oggi per prodotti dal design altamente
ricercato (i ``termoarredi''), non che dall'elevato contenuto tecnologico, quali
impianti VMC, radiatori elettrici connessi, e sistemi di gestione remota di impianti di riscaldamento.

IOTINGA ha aiutato IRSAP nella realizzazione proprio di queste ultime categorie di
prodotti, grazie alla realizzazione della piattaforma ``IRSAP NOW'', un cloud
al quale è possibile connettere tutti i prodotti smart del catalogo IRSAP così da controllarli
da remoto mediante un'applicazione mobile iOS/Android, sempre sviluppata da IOTINGA,
nonché tramite assistenti vocali quali Alexa o Google Home.

All'interno di questa piattaforma si innesta il prodotto oggetto di questa tesi,
ovvero sia la gamma di radiatori elettrici connessi IRSAP. Il catalogo si compone di
decine di prodotti, uno su tutti il ``Polygon'', vincitore del ``CES Best of Innovation 2022'',
nonché di numerosi altri premi a livello internazionale, grazie al suo design
altamente ricercato ed al suo contenuto tecnologico, a cui noi di IOTINGA abbiamo contribuito.

Tale dispositivo è dotato internamente di elettronica in grado di connettersi mediante
Wi-Fi direttamente al cloud ``IRSAP NOW'', ed integra oltre alla funzione scaldante
anche un'illuminazione ambientale mediante led RGBW.

Come IOTINGA abbiamo seguito direttamente lo sviluppo del firmware eseguito sul
dispositivo stesso, oltre che l'integrazione nel cloud e nell'applicazione mobile esistente.

\subsection{Motivazioni (diffcolta' testing IoT/cloud)}

Uno dei problemi principali da evitare specialmente in questa tipologia di prodotti
sono i problemi che si hanno dall'utente finale, per una serie di motivi:

\begin{itemize}
\item la prima impressione sul cliente è quella che conta, se il cliente si ritrova
    un prodotto che funziona male ne parlerà male e creerà un danno d'immagine all'azienda
\item quando il problema si verifica dal cliente è difficile da diagnosticare. I clienti,
    e spesso anche gli installatori stessi, non hanno competenze tecniche e pertanto l'assistenza è difficile
\item se il problema non è risolvibile mediante un aggiornamento firmware è necessario
    effettuare un reso, che ha dei costi molto elevati per il produttore, in quanto
    durante il trasporto molto spesso il prodotto viene danneggiato e quindi deve essere rimpiazzato con un nuovo
\item il prodotto svolge comunque una funzione critica per l'utente, in quanto un suo
    malfunzionamento potrebbe voler dire rimanere senza riscaldamento, con tutti i disagi
    ed i danni che questo può comportare
\end{itemize}

Viene quindi da se che è di fondamentale importanza assicurarsi di far sì che quanti
più problemi possibili vengano identificati prima che il software arrivi nelle mani
dell'utente finale.

Inoltre, visto che la produzione di software esente completamente da bug
è matematicamente impossibile, è fondamentale garantire che i prodotti una volta
installati possano sempre essere aggiornati, in modo da poter rilasciare correzioni agli
inevitabili (e sperabilmente pochi) problemi segnalati dagli utenti.

Tuttavia l'invio di aggiornamenti dovrebbe essere limitata allo stretto indispensabile:
per quante misure di protezione si possano prendere l'operazione di aggiornamento è sempre
molto delicata, in quanto vi sono numerose cose che possono andare storte, ad esempio
può venire a mancare l'alimentazione durante una fase critica dell'aggiornamento,
comportando una corruzione della meoria flash che rende il dispositivo inutilizzabile.

\subsection{Prassi attuale testing (problemi aperti)}

Quando detto al capitolo precedente fa capire quanto sia fondamentale testare in maniera
ancora più approfondita (rispetto a quanto viene attualmente fatto per l'applicazione
mobile ed il cloud) il firmware di questi radiatori.

Attualmente vi sono tre fasi di test:
\begin{enumerate}
    \item test da parte dello sviluppatore
    \item test di accettazione interna (in IOTINGA)
    \item test di accettazione del cliente finale (IRSAP)
\end{enumerate}

\subsubsection{Test di sviluppatore}

Quando uno sviluppatore finisce di implementare una nuova funzionalità o risolve un
bug all'interno del firmware prima di considerare l'attività conclusa ed integrare il
proprio lavoro nel ramo di sviluppo principale ed effettua i propri test.

Questi si occupano sia di verificare che quanto è stato implementato è conforme
alla specifica approvata dal cliente (nel caso di nuove funzionalità) oppure che
il bug sia stato risolto, sia che non siano stato modificato il funzionamento del sistema
nelle parti che sono state impattate dalla modifica.

Tali test sono a discrezione dello sviluppatore, che avendo modificato il codice sa
quali comportamenti sono impattati dalla modifica che ha realizzato e quindi devono essere
provati.

\subsubsection{Test di accettazione interna}

Questi test sono effettuati prima di ogni rilascio di un nuovo artefatto verso il cliente.

Si occupano di validare che il software garantisca il funzionamento di una serie di casi d'uso critici,
senza i quali il sistema stesso non sarebbe utilizzabile e quindi ulteriormente testabile.
Solo se una versione del software passa tutti questi test può essere consegnata al cliente.

Essi si pongono dal punto di vista dell'utente finale, pertanto sono effettuati su un
hardware completo, isolato però dal resto del sistema, ovvero dalla componente cloud
e dall'applicazione mobile. Questo per evitare che vi sia il dubbio che il bug sia
nel cloud o nella app anziché nel dispositivo stesso.

Attualmente vengono effettuati seguendo un documento contenente una serie di passi, ognuno
idicante:

\begin{itemize}
    \item azione: un'operazione da effettuare sul sistema mediante l'interazione fisica
        con il dispositivo (pressione di pulsanti) oppure mediante cloud (tramite un'apposito
        strumento che consente di simulare i messaggi inviati dal cloud e dalla app)
    \item risultato atteso: postcondizioni da verificare dopo aver effettuato l'azione, ad
        esempio ``i led sono rossi'' oppure ``entro 5 secondi viene inviato al cloud un messaggio''.
        Nel caso la postcondizione sia verificata è possibile procedere al passo successivo, altrimenti
        il test viene interrotto con esito negativo, e deve essere segnalato ad uno sviluppatore il problema.
\end{itemize}

Preferibilmetne questa procedura viene fatta eseguire da chi non ha preso parte allo
sviluppo del sistema stesso. Questo per evitare che chi esegue la procedura, conoscendo
le logiche interne del software, possa essere portato a saltare o ignorare determinati
passaggi in quanto ``ovvi'', mentre chi non conosce il sistema è più propenso a seguire
i passaggi alla lettera e segnalare ogni singolo comportamento discordante con quanto
atteso.

L'eseguire la procedura citata sopra tuttavia non è esente da problematiche:

\begin{itemize}
    \item tempo: la procedura richiede del tempo (all'incirca un ora) per essere eseguita.
        Nel caso il test non abbiano successo e si trovi un bug dopo averlo risolto è necessario
        rieseguire i test da capo.
    \item possibilità di errori: essendo una procedura meccanica e ripetitiva è fafile
        introdurre un errore umano nel proprio svolgimento.
\end{itemize}

Oer ovviare a questi problemi andremo a vedere come è stato sviluppato un sistema in
grado di eseguire queste operazioni automaticamente ed in maniera integrata nel flusso
di CI (Continuos Integration) attualmente già presente in azienda.

\subsubsection{Test di accettazione del cliente finale}

Sul cliente finale ricade la responsabilità (anche a livello legale) del prodotto
che viene immesso sul mercato con il proprio nome sopra, e questo include anche
il software. Questo comporta il fatto che a sua volta deve svolgere dei test per
assicurarsi che il software sia conforme a quanto atteso, e nel caso segnalare i
problemi riscontrati in maniera tale che vengano corretti.

Questi test sono volti a testare tutte le funzionalità del prodotto in tutte le loro
possibili configurazioni, anche mediante l'ausilio di strumentazione altamente
specializzata quali camere climatiche per valutarne l'efficacia di termoregolazione.

Nel caso questi test abbiano successo l'artefatto testato passa da stato di candidato
al rilacio a produzione, e viene quindi installato in fabbrica su tutti i nuovi radiatori
prodotti, nonché viene lanciato un aggiornamento OTA su tutti i dispositivi già installati
presso i clienti finali.

Questi test, essendo svolti dal cliente finale, sono fuori dai nostri scopi di analisi.

\section{Related work}
\subsection{Tool commerciali/open-source}

\section{Approccio}
\subsection{Panoramica}

Andiamo a vedere come è realizzato il dispositivo in oggetto di questa tesi.

L'elettronica viene prodotta in due modelli differenti:
\begin{itemize}
    \item luxury: versione basilare della scheda elettronica, installata tipicamente
        sui prodotti da bagno (scaldasalviette)
    \item design: elettronica riservata ai prodotti di fascia più alta. Offre in
        aggiunta a tutto quanto offerto dalla precedente elettronica un sensore VOC
        (qualità dell'aria) aggiuntivo e la possibilità di collegare una striscia a
        led RGBW per l'illuminazione ambientale. Inoltre è progettata in maniera da
        integrarsi direttamente all'interno del radiatore, di modo che non sia direttamente
        visibile.
\end{itemize}

Inoltre, un radiatore a parità di elettronica può o meno avere a disposizione le
seguenti funzionalità:
\begin{itemize}
    \item Fil Pilote: uno standard francese per l'interconnessione dei radiatori ad una
        centralina di controllo mediante un secondo ingresso di segnale a 230V
    \item led RGBW: disponibili solo su radiatore con scheda design, consente di
        illuminare l'ambiente circostante al radiatore per creare atmosfera
\end{itemize}

Tutte le versioni di scheda elettronica eseguono la stessa versione del firmware,
che ha una fase iniziale in cui identifica la tipologia di scheda e le funzionalità
opzionali abilitate, e di conseguenza configura le periferiche del microcontrollore di conseguenza.

Tuttavia per essere esaustivi è necessario svolgere i test di accettazione su entrambe
le versioni di scheda elettronica, in quanto possono esservi comportamenti differenti.

\subsection{Caratteristiche hardware}

La piattaforma hardware ruota attorno ad un unico microcontrollore Wi-Fi della Telit, il GS2200M.
Le caratteristiche di questo chip sono notevoli:

\begin{itemize}
    \item processore dual core ARM Cortex M3 fino a 120Mhz, di cui un core dedicato
        alla gestione del Wi-Fi ed uno all'esecuzione dell'applicativo
    \item 1Mb di RAM in totale, di cui all'incirca 500kB utilizzabile dall’applicazione,
        il resto dedicata all'uso della parte Wi-Fi
    \item 4Mb di memoria flash interna, in parte dedicata al codice del firmware ed
        in parte come filesystem interno in cui memorizzare i dati dell'applicazione
    \item interfaccia Wi-Fi b/g/n a 2.4Ghz in grado di operare sia in modalità station
        (client) sia che access-point (AP) a cui connettersi direttamente
    \item 19 input/output digitali
    \item 3 output PWM
    \item 2 ingressi analogici mediante ADC, uno a 10 ed uno a 12 bit
    \item un interfaccia I2C hardware
    \item un interfaccia SPI hardware
    \item due interfacce seriali UART
    \item modulo RTC interno
\end{itemize}

Tale microcontrollore si interfaccia con le seguenti periferiche hardware presenti
sull'elettronica:

\begin{itemize}
    \item una resistenza (cartuccia) a 230V che provvede a riscaldare il fluido
        termoconduttivo presente all'interno del radiatore
    \item una sonda di temperatura NTC usata per rilevare la temperatura ambiente
    \item una pulsantiera dotata di due pulsanti capacitivi e dei led RGB di illuminazione
        come feedback verso l'utente
    \item un buzzer utilizzato per dare un feedback uditivo all'utente
    \item solo per le schede ``design'' un sensore di qualità dell'aria in grado di
        misurare i livelli VOC e CO2
    \item solo per i modelli che lo prevedono un ingresso per il segnale ``Fil Pilote''
    \item solo per i modelli che li prevedono una striscia a led RGBW di illuminazione ambientale
    \item sempre solo per alcuni modelli una seconda sonda di temperatura in grado di
        misurare la temperatura del corpo riscaldante, in maniera tale da migliorare
        l'accuratezza degli algoritmi di termoregolazione
\end{itemize}

\subsection{Caratteristiche software}

Il firmware che gira sul dispositivo è scritto in linguaggio C99. La piattaforma
Telit mette a disposizione un ambiente di sviluppo basato sul compilatore IAR C
compiler.

L'SDK fornito dal produttore include il RTOS ThreadX e la relativa
implementazione dello stack di rete NetX, oltre che una serie di librerie per:

\begin{itemize}
    \item connessione e configurazione del Wi-Fi
    \item filesystem interno simil-FAT
    \item connessioni TLS
    \item server e client HTTP
    \item aggiornamento OTA con doppia partizione
\end{itemize}

Il firmware è essenzialmente suddiviso in 3 moduli:

\begin{itemize}
    \item gestione della termoregolazione, che si occupa di tutto quel che è necessario
        per regolare la temperatura ambiente secondo le modalità di funzionamento del dispositivo
    \item gestione della comunicazione cloud, si occupa di mantenere sincronizzato lo stato
        interno del dispositivo con il server ``IRSAP NOW'' mediante protocollo MQTT
    \item gestione dell'interfaccia utente, si occupa di rispondere agli input dell'utente
        sulla pulsantiera e di darne relativo feedback mediante l'uso dei LED RGB e del buzzer
\end{itemize}

\subsection{Stati interni del dispositivo}

Il dispositivo ad alto livello può trovarsi in 3 stati distinti:

\begin{itemize}
    \item non abbinato: in attesa di un primo abbinamento da app. Ogni funzione del
        dispositivo è esclusa finché l'utente non lo collega mediante l'applicazione
    \item non connesso: il dispositivo è stato in passato abbinato ma al momento non è
        connesso al cloud, perché ad esempio la connessione Wi-Fi non è momentaneamente disponibile
    \item connesso: il dispositivo è connesso e sincronizzato con il cloud
\end{itemize}

\begin{figure}[ht]
    \centering
    \begin{tikzpicture}
        \node[state] (unbounded) {Non abbinato};
        \node[state, below left of=unbounded, below=1cm] (offline) {Non connesso};
        \node[state, right of=offline] (online) {Connesso};
        \draw (unbounded) edge[bend left, align=left, right] node{abbinato\\con successo} (online)
        (offline) edge[bend left, above] node{connessione} (online)
        (offline) edge[bend left, left, align=right] node{ripristino\\di fabbrica} (unbounded)
        (online) edge[bend left, below] node{disconnessione} (offline)
        (online) edge[bend left, align=right, right] node{ripristino\\di fabbrica} (unbounded);
    \end{tikzpicture}
    \caption{Stati del radiatore}
    \label{stati}
\end{figure}

\subsection{Termoregolazione}

La componente di termoregolazione si occupa di regolare la temperatura ambiente portandola
il più vicino possibile a quanto desiderato dall'utente (set-point) mediante
il controllo dell'accensione (on/off) dell'elemento riscaldante. Il feedback sulla
temperatura ambiente è ottenuto mediante la sonda di temperatura NTC.

Il dispositivo ha diversi modi di funzionamento:
\begin{itemize}
    \item standby: dispositivo completamente spento, sia per quanto riguarda il riscaldamento
        che per l'illuminazione LED
    \item antigelo: il dispositivo mantiene una temperatura di sicurezza (impostata dall'utente)
        per prevenire danni dati da una tempratura ambiente troppo bassa (ad es. congelamento delle tubature)
    \item vacanza: all'interno di un intervallo temporale impotato dall'utente funziona
        in modalità antigelo
    \item away: imposta un set-point ridotto (ECO) in quanto l'utente non è in casa
    \item programmato: segue una programmazione settimanale che consente per ogni
        giorno della settimana di creare fino ad 8 fasce orarie
    \item manuale temporaneo: segue un set-point manuale per un determinato tempo
    \item manuale: segue il set-point utente che è fisso e non varia mai
        configurato dall'utente, quindi torna a funzionare nella modalità precedente
    \item fil pilote: il dispositivo è controllato (ove disponibile) da un segnale
        esterno in ingresso sul cavo fil pilote
\end{itemize}

È possibile mediante interfaccia utente muoversi fra le varie modalità come dettagliato
in figura \ref{modi}.

\begin{figure}[ht]
    \centering
    \begin{tikzpicture}
        \node[state] (programmato) {Programmato};
        \node[state, right of=programmato, align=center] (manuale-tempo) {Manuale\\temporaneo};
        \node[state, right of=manuale-tempo] (manuale) {Manuale};
        \draw (manuale) edge[loop above] node{modifica set-point} (manuale)
            (programmato) edge[bend left, above] node{modifica set-point} (manuale-tempo)
            (manuale-tempo) edge[bend left, above] node{modifica set-point} (manuale);
    \end{tikzpicture}
    \caption{Modi del radiatore}
    \label{modi}
\end{figure}

\subsection{Comunicazione cloud}

La componente cloud è implementata su AWS. La connessione avviene grazie al protocollo
MQTT usando il servizio broker di AWS IoT Core.

La connessione è cifrata mediante TLS ed autenticata mediante certificato TLS client
specifico per il singolo dispositivo, generato e programmato nel dispositivo in fase
di produzione della scheda elettronica da una CA intermedia a sua volta firmata dalla
CA root di AWS come in figura \ref{tls-chain}.

\begin{figure}[ht]
    \centering
    \begin{tikzpicture}
        \node[rectangle, draw, minimum height=1cm] (root) {root CA AWS};
        \node[rectangle, draw, right of=root, align=center, minimum height=1cm] (intermedia) {CA intermedia\\ produttore};
        \node[rectangle, draw, right of=intermedia, minimum height=1cm] (dispositivo) {certificato dispositivo};
        \draw (root) edge[above] node{firma} (intermedia)
            (intermedia) edge[above] node{firma} (dispositivo);
    \end{tikzpicture}
    \caption{Catena TLS}
    \label{tls-chain}
\end{figure}

Per quanto concerne il protocollo di comunicazione abbiamo scelto di adottare un
protocollo binario, in quanto l'utilizzo del JSON poneva sia delle problematiche
di gestione di memoria all'interno del dispositivo, sia andava ad appesantire il
payload trasferito andando a generare potenziali costi su cloud altrimenti evitabili.

Il protocollo è stateless, nel senso che viene usato per trasferire stati completi
del dispositivo fra server e client. Vengono definite 3 operazioni fondamentali
che è possibile compiere con il protocollo:

\begin{itemize}
    \item GET: richiesta dello stato corrente
    \item PUT: aggiornamento dello stato corrente
    \item DELETE: eliminazione dello stato corrente
\end{itemize}



\section{Implementazione}
\subsection{Interfacce}
\subsection{Implementazione hardware}
\subsection{Implemnetazione software}
\subsection{Integrazione (continuous integration)}

\section{Validazione sperimentale}
\subsection{scenari da testare}
\subsection{tempi di esecuzione (confronto automatico/manuale)}

\section{Comclusioni}
\subsection{Considarzioni}
\subsection{Passi futuri}

\end{document}
